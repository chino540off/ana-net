\usepackage[english]{babel}
\usepackage[utf8]{inputenc}
\usepackage[T1]{fontenc}
\usepackage{type1cm}
\usepackage{ae,aecompl}
\usepackage{a4}
\usepackage{graphicx}
\usepackage[font=small,labelfont=bf,justification=RaggedRight]{caption}
\usepackage{subfigure}
\usepackage{fancyhdr}
\usepackage{fancybox}
\usepackage{pst-gantt}
\usepackage{float}
\usepackage{longtable}
\usepackage{paralist}
%%\usepackage{hyperref}
\usepackage{url}
% Allow better control over tabular environments
\usepackage{array}
% Algorithm environment based on algorithmicx.sty.
% No need to include algorithmicx
% Include algorithm and one of {algpseudocode, algpascal, algc}
%\usepackage{algorithmicx} % Do not include algorithmicx
\usepackage{algorithm}
\usepackage{algpseudocode}
%\usepackage{algpascal}
%\usepackage{algc}
% URL handling
\usepackage{url}
\usepackage{amsmath}
\usepackage{amsfonts}
% Font for listings
%\usepackage{LuxiMono}
% Program listings
\usepackage[final]{listings}
\lstloadlanguages{[GNU]C++,Perl,tcl}
\lstset{language=[GNU]C++}
\newcommand{\?}{\discretionary{/}{}{/}}
%\newcommand{\liter}[0]{/home/ruf/Lib/Bibl/}
\newcommand{\reffig}[1]{\mbox{Figure~\ref{#1}}}
\newcommand{\reftbl}[1]{\mbox{Table~\ref{#1}}}
\newcommand{\refsec}[1]{\mbox{Section~\ref{#1}}}
\newcommand{\refchp}[1]{\mbox{Chapter~\ref{#1}}}
\newcommand{\refapp}[1]{\mbox{Appendix~\ref{#1}}}
\newcommand{\refalg}[1]{\mbox{Algorithm~\ref{#1}}}
\newcommand{\reflst}[1]{\mbox{Listing~\ref{#1}}}
\newcommand{\refeqn}[1]{\mbox{\eqref{#1}}}
\pagestyle{fancy}
%%-lpr Note: 'chapters' are defined for 'book's only
%%-lpr       in articles, we make use of sections only
%%-lpr \renewcommand{\chaptermark}[1]{\markboth{#1}{}}
\renewcommand{\chaptermark}[1]{\markboth{\thechapter\ #1}{}}
\renewcommand{\sectionmark}[1]{\markright{\thesection\ #1}}
\fancyhf{}
\fancyhead[LE,RO]{\bfseries\thepage}
\fancyhead[LO]{\bfseries\rightmark}
\fancyhead[RE]{\bfseries\leftmark}
\renewcommand{\headrulewidth}{0.2pt}
\addtolength{\headheight}{0.5pt}
\fancypagestyle{plain}{%
   \fancyhf{}
   \fancyfoot[C]{\bfseries \thepage}
   \fancyhead{}%get rid of headers on plain pages
	 \renewcommand{\headrulewidth}{0pt} % get rid of the line on plain pages
}
\newcommand{\clearemptydoublepage}{\newpage{\pagestyle{empty}\cleardoublepage}}
% Indent paragraphs by 1em
\setlength{\parindent}{1em}
\newcommand{\Appendix}[2][?]
{
  \refstepcounter{section}
  \addcontentsline{toc}{appendix}
  {
    \protect\numberline{\appendixname~\thesection} %1
  }
  {
    \flushright\large\bfseries\appendixname\ \thesection\par
    \nohypens\centering#1\par
  }
  \vspace{\baselineskip}
}
% Stressed words
\newcommand\strong[1]{\textbf{#1}}
% First occurrence of a key word
\newcommand\keyfirst[1]{\textbf{#1}}
% Title in an eunmerate environment
\newcommand\enumtitle[1]{\textbf{#1}}
% Leave one empty line before paragraph
\newcommand\largeskip{\vspace{\baselineskip}\noindent}
% Bibliography
% Number like [23]
\bibliographystyle{ieeetr}
% Number like [ZKF02]
%\bibliographystyle{alpha}
% Add line to table of contents
\usepackage[notlot,nottoc,notlof]{tocbibind}
