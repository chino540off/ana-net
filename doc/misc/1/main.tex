\documentclass[times,10pt,twocolumn]{article} 
\usepackage{main}
\usepackage{times}
\usepackage[english]{babel}
\usepackage{hyperref}
\usepackage{listings}
\usepackage[utf8]{inputenc}
\usepackage{latexsym}
\pagestyle{empty}

\parindent0em

\begin{document}
\title{
Ideas on context-oriented networking on LANA\\\smallskip [DRAFT]
}

\author{
Daniel Borkmann\\
dborkma@tik.ee.ethz.ch\\
}

\maketitle
\thispagestyle{empty}

\begin{abstract}
This document includes notes about possible context-oriented networking 
LANA (Lightweight Autonomic Network Architecture) regarding API design, 
addressing, routing, stack setup and more; rather loosely organized and 
(for now) more in the sense of expressing an idea than having a full 
sophisticated specification.
\end{abstract}

\Section{Addressing Scheme}
\label{addr}
The addressing scheme which is used by userland applications must be
independent of IPv4 or IPv6 and should be very generic and simple.
However, it should also fit into the BSD socket API. A \texttt{communication 
endpoint} in LANA is therefore defined as
\begin{quote}\texttt{peer}@\texttt{[segment]}\end{quote}
where
\begin{itemize}
	\setlength{\itemsep}{-1mm}
	\item \texttt{peer}$\in\mathcal{L}(peer)$ with arbitrary length
	\item \texttt{[segment]} is a string concatenation of
	      \texttt{segment} separated by the character \texttt{.} (dot) to the right
	\item \texttt{segment}$\in\mathcal{L}(segment)$ with arbitrary length
\end{itemize}
and
\begin{itemize}
	\setlength{\itemsep}{-1mm}
	\item $\Sigma_{1} =\{a,b,c,\ldots,z,0,1,2,\ldots,9\}$
	\item $\Sigma_{2} =\{$-$\}$
	\item $\Sigma_{3} =\{.\} \cup \Sigma_{2}$
	\item $\mathcal{L}(peer)=\{\alpha^{+}(\beta\alpha^{+})^{*}\} \;\vert\; \alpha\in\Sigma_{1},\beta\in\Sigma_{3}$
	\item $\mathcal{L}(segment)=\{\alpha^{+}(\beta\alpha^{+})^{*}\} \;\vert\; \alpha\in\Sigma_{1},\beta\in\Sigma_{2}$
\end{itemize}

The concatenation of \texttt{segment}s forms a namespace like in 
\textit{Java} package namespaces for instance, but with a different
semantic. So the more \texttt{segment}s are concatenated to the 
right, the more fine-grained networking \texttt{peer group}, in 
the sense of 'all \texttt{peer}s within a single \texttt{[segment]}', 
we have.\newline

An empty \texttt{[segment]} is not allowed, there must be at 
least one \texttt{segment} and the dot character after the most
fine-grained \texttt{segment} must be left out.\newline

Examples of valid \texttt{communication endpoint}:
\begin{itemize}
	\setlength{\itemsep}{-1mm}
	\item \texttt{sensor42@csg.temp.floor1}
\end{itemize}

Furthermore, there are three special \texttt{peer}s:\newline

A \texttt{peer} called \texttt{*} which defines a broadcast \texttt{peer}, 
so that a message sent to \texttt{*}@\texttt{[segment]} will be received by
the whole \texttt{peer group} of the \texttt{[segment]}.\newline

Next to this, there are also multicast \texttt{peer}s, which have an address of 
form \texttt{+peer}@\texttt{[segment]}, thus the multicast group name
language is in the same $\mathcal{L}$ as $\mathcal{L}(peer)$, but with a
\texttt{+} (plus) as prefix.\newline

And the last special \texttt{peer} is used for anycasts, which form an 
address of \texttt{-peer}@\texttt{[segment]}. Similar to multicast group 
names, they have a \texttt{-} (minus) instead of \texttt{+} (plus) as 
prefix.\newline

Examples of special \texttt{communication endpoint}s:
\begin{itemize}
	\setlength{\itemsep}{-1mm}
	\item \texttt{*@csg.temp.floor1} (broadcast)
	\item \texttt{+sensors@csg.temp.floor1} (multicast)
	\item \texttt{-controller@csg.temp.floor1} (anycast)
\end{itemize}

However, the regular \texttt{communication endpoint}, a concrete instance of 
\texttt{peer}@\texttt{[segment]}, forms an unique communication address and 
is represented by a BSD socket object, thus one host can have applications
running that open sockets and communicate to others. In this addressing scheme
a \texttt{peer} can be seen (in traditional TCP/IP network architectures) as 
a combination of an IP address within the \textit{current} LAN \textit{and} 
a port associated to it that offeres or uses a service. The \texttt{[segment]} 
part of the addressing scheme can be seen as a specific subnet the \texttt{peer}
belongs to. So by this, one host has always the same \texttt{[segment]},
and may have different \texttt{peer}s depending on its applications.\newline

The \texttt{[segment]}s namespace should be well-structured and
well-defined - similar as DNS for instance, but in this case to describe
subnets. Note that \texttt{csg.temp.floor1} is a different subnet as
\texttt{csg.temp}, thus routing is needed to communicate
with \texttt{peer1}@\texttt{csg.temp.floor1} and \texttt{peer2}@\texttt{csg.temp}. 
Also, one and the same \texttt{peer} may be in different \texttt{[segment]}s, 
i.e.
\begin{itemize}
	\setlength{\itemsep}{-1mm}
	\item \texttt{sensor42}@\texttt{csg.temp.floor1}
	\item \texttt{sensor42}@\texttt{csg.temp.floor2}
\end{itemize}

With the syntax and semantic of \texttt{peer}@\texttt{[segment]},
IP and port numbers become redundant (conjecture 1).\newline

Now, since both the \texttt{peer} and the \texttt{[segment]} can be strings 
according to their given $\mathcal{L}$ with arbitrary length, the underlying
operating system and hardware must have a proper way of handling such 
strings.\newline

Therefore, the underlying representation to the kernel looks like:
\begin{itemize}
	\setlength{\itemsep}{-1mm}
	\item \underline{\texttt{peer}} $\rightarrow$ \texttt{SHA1(peer)}
	\item \underline{\texttt{[segment]}} $\rightarrow$ \texttt{SHA1([segment])}
\end{itemize}

Thus we have now a fixed-length representation of
\underline{\texttt{peer}}@\underline{\texttt{[segment]}} of $2*20$ Byte per
\texttt{communication endpoint}. We \textit{assume} that \texttt{SHA1} is
safe enough to use, so that there is no chance to have a kollision of a pair
of different \texttt{peer}s and at the same time a kollision of a pair of
different \texttt{[segment]}s (conjecture 2).\newline

The benefit from this addressing scheme is that we will not ran out 
of addresses respectively \texttt{peer}s as in IPv4 for instance, thus 
there would be no need to have such a long-time transistion to IPv6.
Organizational entities can dynamically reorganize their \texttt{[segment]} 
structure and we do not ran out of \texttt{[segment]}s, too. Furthermore,
the addressing scheme is more context oriented and consists not just of
'meaningless numbers'.

\Section{Basic Protocol Types}
With this addressing scheme we have two basic protocol layers on top of the 
underlying hardware:
\begin{itemize}
	\setlength{\itemsep}{-1mm}
	\item Hardware address layer (can be MAC or similar)
	\item \texttt{peer}@\texttt{[segment]} layer
\end{itemize}

Furthermore, we have three basic protocol types on top of an underlying 
carrier like Ethernet, InfiniBand or Bluetooth:
\begin{itemize}
	\setlength{\itemsep}{-1mm}
	\item \texttt{ETH\_P\_LACM} $\rightarrow$ Intra-segment config messages
	\item \texttt{ETH\_P\_LADM} $\rightarrow$ Intra-segment data messages
	\item \texttt{ETH\_P\_LRDM} $\rightarrow$ Inter-segment data messages
\end{itemize}

\texttt{ETH\_P\_LACM} is for name resolution, route resolution and other 
\texttt{peer}@\texttt{[segment]} management messages. \texttt{ETH\_P\_LADM} is for data 
messages within a given \texttt{[segment]} thus the 
\underline{\texttt{[segment]}} must \textit{not} be encoded into the packet. 
\texttt{ETH\_P\_LRDM} is for data messages between different \texttt{[segment]}s, 
so that the source and destination \underline{\texttt{[segment]}}
hashes must be encoded into the packet, since these packets need
routing.\newline

Concrete identifiers for \texttt{ETH\_P\_LACM}, \texttt{ETH\_P\_LADM} and
\texttt{ETH\_P\_LRDM} must be choosen in a way, that there is still compatiblity
respectively no interference of traditional Internet protocols such as IPv4, 
IPv6 and others.

\Section{Host Information Management}
Since - as stated in \ref{addr} - one host can have multiple \texttt{peer}s, 
there are two kernelspace \texttt{peer}@\texttt{[segment]}s per NIC with a 
\underline{\texttt{peer}} each that has a truly randomly assigned value 
(i.e. via \texttt{/dev/random}). These two \texttt{peer}s are used for a hosts 
information management, conrete:
\begin{itemize}
	\setlength{\itemsep}{-1mm}
	\item \texttt{peer} to hardware address resolution of
              \texttt{peer}s within the own \texttt{[segment]}
	\item Inter-\texttt{[segment]} route resolution
\end{itemize}

Both \texttt{peer}@\texttt{[segment]}s may only exchange messages within the
\texttt{ETH\_P\_LACM} protocol.\newline

Now for each of these both special kernelspace \texttt{peer}s, a distributed
hash table (DHT) is assigned
\begin{itemize}
	\setlength{\itemsep}{-1mm}
	\item $f:$ \underline{\texttt{peer}}$\rightarrow$hardware 
              address of \texttt{peer}
	\item $g:$ \underline{\texttt{[segment]}}$\rightarrow$hardware 
              address of \texttt{peer} that is a gateway to a different 
              \texttt{[segment]} and has a route to this destination
\end{itemize}
so that the information respectively mapping of $f$ and $g$ is autonomously 
managed within the \texttt{peer group} of this \texttt{[segment]}. By this,
no broadcast on the hardware address layer is needed at all (conjencture 3),
thus \texttt{[segment]} may contain more participants as typical subnets of 
a traditional LAN.\newline

The DHT information management from Kademlia \cite{PeterMaymounkovDavidMazieres} 
is applied, but with the exception that messages are exchanged on the hardware
address layer instead of traditional IP. The identifier, which is used in the
Kademlia XOR routing metric will be the specific kernelspace 
\underline{\texttt{peer}}.\newline

A resolution of the hardware address of \texttt{peer}s within a 
\texttt{[segment]} then equals a lookup in the DHT of function $f$ on the 
hardware address layer. Similar, the resolution of a route equals a lookup 
in the DHT of function $g$ on the hardware address layer.\newline

Lookups in the DHT are performed in $\mathcal{O}(\log{n})$ where $n$ is the number
of \texttt{peer}s of the hosts \texttt{[segment]}. Local caches can optinally 
be applied, so that a lookup in the DHT of $f$ or $g$ does not need to be 
performed twice within a given period.\newline

Now if a host has multiple NICs and therefore multiple kernelspace 
\texttt{peer}s, the kernel must of course be able to access the DHTs 
of the other NICs.

\Section{Routing between \texttt{[segment]}s}
\label{route}
As mentioned previously, the communication of \texttt{peer}s within a
\texttt{peer group} involves no routing at all, so this is not further
being discussed within this section. Also, asking for a route always 
equals a DHT lookup on $g$ within the hosts \texttt{[segment]}.\newline

The DHT for the function $g$ can only be filled, if a host has at least
two NICs connected to different \texttt{[segment]}s. This host then publishes
its possible routes to its connected \texttt{[segment]}s into the corresponding 
DHTs $g$.\newline

If for instance a host $x$ with two NICs $NIC_a$, $NIC_b$ is connected to two 
different \texttt{[segment]}s $S_a$, $S_b$, it announces to the DHT $g$ 
of $S_b$ via $NIC_b$ that a route to $S_a$ is reachable from $NIC_b$ and 
vice versa to the DHT $g$ of $S_a$ via $NIC_a$ that a route to $S_b$ is
reachable from $NIC_a$, so that others can simply perform a lookup to their
\texttt{[segment]}s DHT $g$ to obtain the related NICs hardware address 
of $x$ and then forward data packets of type \texttt{ETH\_P\_LRDM}.\newline

Now this mechanism works fine for routing between neighboring 
\texttt{[segment]}s. However, what about routing between \texttt{[segment]}s
that are not directly connected to each other?\newline

One possibility would be to have transversely routes next to the 
\texttt{[segment]}s tree structure. The transversely routes could then be
regarded as a way of skipping the usual communication path through the tree
by having a shorter path and therefore to reduce network load on the trees
upper and root nodes. This structure could be seen similar to a 
\textit{skip list} as some form of a \textit{skip tree}. By having lots of 
such transversely routes the network structure is then being meshed by
having a worst case route of traversing the \texttt{[segment]}s tree.\newline

A worst case routing example is provided next. Here, \texttt{peer1@csg.temp.floor1}
wants to communicate with \texttt{peer2@csg.staff.bureau1}. Since in the worst
case there are no transversely routes i.e. directly from \texttt{csg.temp.floor1} to 
\texttt{csg.staff.bureau1}, we need to traverse the tree to its root \texttt{[segment]} 
named \texttt{csg} and then downwards to \texttt{csg.staff.bureau1}:\newline

Lookup 1 from \texttt{csg.temp.floor1}:
\begin{enumerate}
	\setlength{\itemsep}{-1mm}
	\item \texttt{csg.staff.bureau1}, if failed
	\begin{enumerate}
		\setlength{\itemsep}{-1mm}
		\item \texttt{csg.staff}, if failed
		\begin{enumerate}
			\setlength{\itemsep}{-1mm}
			\item \texttt{csg}, if failed
			\begin{enumerate}
				\setlength{\itemsep}{-1mm}
				\item \texttt{csg.temp}, must succeed, so forward to \texttt{csg.temp}
			\end{enumerate}
		\end{enumerate}
	\end{enumerate}
\end{enumerate}

Lookup 2 from \texttt{csg.temp}:
\begin{enumerate}
	\setlength{\itemsep}{-1mm}
	\item \texttt{csg.staff.bureau1}, if failed
	\begin{enumerate}
		\setlength{\itemsep}{-1mm}
		\item \texttt{csg.staff}, if failed
		\begin{enumerate}
			\setlength{\itemsep}{-1mm}
			\item \texttt{csg}, must succeed, so forward to \texttt{csg}
		\end{enumerate}
	\end{enumerate}
\end{enumerate}

Lookup 3 from \texttt{csg}:
\begin{enumerate}
	\setlength{\itemsep}{-1mm}
	\item \texttt{csg.staff.bureau1}, if failed
	\begin{enumerate}
		\setlength{\itemsep}{-1mm}
		\item \texttt{csg.staff}, must succeed, so forward to \texttt{csg.staff}
	\end{enumerate}
\end{enumerate}

Lookup 4 from \texttt{csg.staff}:
\begin{enumerate}
	\item \texttt{csg.staff.bureau1}, must succeed, so forward to \texttt{csg.staff.bureau1}
\end{enumerate}

With a more meshed network structure, we could have a best case routing for this specific example
of:\newline

Lookup 1 from \texttt{csg.temp.floor1}:
\begin{enumerate}
	\item \texttt{csg.staff.bureau1}, succeeded, so forward to \texttt{csg.staff.bureau1}
\end{enumerate}

\Section{Firewalling / Filtering}
A \texttt{[segment]} can be protected from unwanted traffic by doing a
string pattern matching (i.e. in hardware) on the \texttt{[segment]}
string. Since this is represented as \underline{\texttt{[segment]}} one
can do blacklist or whitelist matching. This is also possible for allwing
traffic for concrete \underline{\texttt{peer}}@\underline{\texttt{[segment]}}
for instance. However, pattern matching in the sense of regular expression
matching will not be possible since \texttt{peer}@\texttt{[segment]} are
represented as \texttt{SHA1} hashes.

\Section{Multicast Groups}
A multicast group is handeled as an usual \texttt{peer} within the DHT and the 
multicast group management is postponed to the hardware address layer, if the harware 
address layer supports multicast. (still needs more explanation)

\Section{Anycast Groups}
An anycast group is handeled different within the DHT than the multicast group. If 
a \texttt{peer} joins an anycast group, then the keys value needs to maintain a list
of \texttt{peer}s associated to a specific given key. One \texttt{peer} is then returned
from the list, for instance in a round-robin or random fashion. (still needs more explanation)

\Section{Communication Bootstrap}
(todo)

\Section{Userspace API}
The application developer interface for the userland is discussed here.

\SubSection{Addressing Structure}
The addressing structure defined in \texttt{linux/lana.h} is named
\texttt{struct sockaddr\_lana} and consists of the elements
\begin{itemize}
	\setlength{\itemsep}{-1mm}
	\item \texttt{sa\_family} which is always \texttt{AF\_LANA} or \texttt{PF\_LANA}
	\item \texttt{sa\_address} which has the form
	      \texttt{peer}@\texttt{[segment]} as described in \ref{addr}
\end{itemize}

\SubSection{Segment Map}
The file \texttt{/etc/segment} must contain the \texttt{[segment]} to NIC 
mapping of all NICs, i.e.:\newline

\scriptsize{
\begin{lstlisting}
eth0:csg.temp.floor1
eth1:csg.temp.admin
\end{lstlisting}
}
\normalsize

This example implies a direct gateway from \texttt{csg.temp.floor1} to 
the \texttt{csg.temp.admin} \texttt{[segment]} and vice versa.

\SubSection{Systemcalls}
Supported BSD socket system calls for LANA are:
\begin{itemize}
	\setlength{\itemsep}{-1mm}
	\item \texttt{socket(2)}
	\item \texttt{bind(2)}
	\item \texttt{getsockopt(2)/setsockopt(2)}
	\item \texttt{select(2)/poll(2)/epoll(2)}
	\item \texttt{recvfrom(2)}
	\item \texttt{sendto(2)}
	\item \texttt{close(2)}
\end{itemize}

Note that the \texttt{bind(2)} system call must be done before any calls
to \texttt{recvfrom(2)} or \texttt{sendto(2)} in order to bind the socket
to a specific \texttt{peer}@\texttt{[segment]} otherwise the kernel will
assign a randomly choosen string for \texttt{peer} and a randomly selected
\texttt{[segment]} of \texttt{/etc/segment}. The latter assumes, that this
socket only wants to \textit{consume} services of other
\texttt{peer}@\texttt{[segment]}s, without offering some service for others.
Compare this to the notion of port numbers, where the kernel randomly
assigns a number for communication on client applications. Then, the kernel
will do internally a bind before a \texttt{recvfrom(2)} or \texttt{sendto(2)}
call. However, if a \texttt{bind(2)} is done explicitly and the \texttt{[segment]}
is different from \texttt{[segment]}s given in \texttt{/etc/segment}, then
the bind will fail and an error is returned.\newline

With \texttt{setsockopt(2)} one is able join or leave multicast and anycast
groups.

\SubSection{Message flags}
Message flags are statical, global and defined in \texttt{linux/lana.h} and 
represented as bitvectors. They are applied to the \texttt{sendto(2)} call, 
thus the kernel gets to know the applications needs for communication with 
the remote \texttt{communication endpoint}, e.g. whether the communication 
must be reliable or confidential. These needs are named \texttt{communication 
features}.

\SubSection{Code Example}
Here is an example C code of how a typical LANA userspace program 
for a temperature measuring entity could look like:\newline

\scriptsize{
\lstset{language=C}
\begin{lstlisting}
#include <linux/lana.h>

int main(void)
{
  ssize_t ret;
  int sock, err, flags;
  struct sockaddr_lana own, remote;
  char buff[256];
  socklen_t remote_len;
  ...
  sock = socket(AF_LANA, SOCK_RAW, 0);
  ...
  own.sa_family = AF_LANA;
  strlcpy(own.sa_address,
          "sensor42@csg.temp.floor1",
          sizeof(own.sa_address));
  bind(sock, (struct sockaddr_in *) &own, sizeof(own));
  ...
  remote_len = sizeof(remote);
  recvfrom(sock, buff, sizeof(buff), 0,
           (struct sockaddr *) remote,
           &remote_len);
  ...
  memset(&remote, 0, sizeof(remote));
  remote.sa_family = AF_LANA;
  strlcpy(remote.sa_address,
          "-controller@csg.temp.floor1",
          sizeof(remote.sa_address));
  memset(buff, 0, sizeof(buff));
  strlcpy(buff, "temp:20,scale:celsius", sizeof(buff));
  flags = MSG_GLOBAL | MSG_RELIABLE;
  sendto(sock, buff, strlen(buff) + 1, flags,
         (struct sockaddr *) &remote, sizeof(remote));
  ...
  memset(&remote, 0, sizeof(remote));
  remote.sa_family = AF_LANA;
  strlcpy(remote.sa_address, "*@csg.temp.floor1",
  	sizeof(remote.sa_address));
  memset(buff, 0, sizeof(buff));
  strlcpy(buff, "alive", sizeof(buff));
  flags = MSG_GLOBAL | MSG_RELIABLE;
  sendto(sock, buff, strlen(buff) + 1, flags,
         (struct sockaddr *) &remote, sizeof(remote));
  ...
  close(sock);
  return 0;
}
\end{lstlisting}
}
\normalsize

\Section{Network Stack Architecture}
\SubSection{Functional Blocks}
\SubSection{Information Dispatch Point (IDP)}
\SubSection{Packet Processing Engine}

\nocite{*}
\bibliographystyle{abbrv}
\bibliography{main}

\end{document}
